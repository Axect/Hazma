%% Generated by Sphinx.
\def\sphinxdocclass{report}
\documentclass[letterpaper,10pt,english]{sphinxmanual}
\ifdefined\pdfpxdimen
   \let\sphinxpxdimen\pdfpxdimen\else\newdimen\sphinxpxdimen
\fi \sphinxpxdimen=.75bp\relax

\usepackage[utf8]{inputenc}
\ifdefined\DeclareUnicodeCharacter
 \ifdefined\DeclareUnicodeCharacterAsOptional
  \DeclareUnicodeCharacter{"00A0}{\nobreakspace}
  \DeclareUnicodeCharacter{"2500}{\sphinxunichar{2500}}
  \DeclareUnicodeCharacter{"2502}{\sphinxunichar{2502}}
  \DeclareUnicodeCharacter{"2514}{\sphinxunichar{2514}}
  \DeclareUnicodeCharacter{"251C}{\sphinxunichar{251C}}
  \DeclareUnicodeCharacter{"2572}{\textbackslash}
 \else
  \DeclareUnicodeCharacter{00A0}{\nobreakspace}
  \DeclareUnicodeCharacter{2500}{\sphinxunichar{2500}}
  \DeclareUnicodeCharacter{2502}{\sphinxunichar{2502}}
  \DeclareUnicodeCharacter{2514}{\sphinxunichar{2514}}
  \DeclareUnicodeCharacter{251C}{\sphinxunichar{251C}}
  \DeclareUnicodeCharacter{2572}{\textbackslash}
 \fi
\fi
\usepackage{cmap}
\usepackage[T1]{fontenc}
\usepackage{amsmath,amssymb,amstext}
\usepackage{babel}
\usepackage{times}
\usepackage[Bjarne]{fncychap}
\usepackage[dontkeepoldnames]{sphinx}

\usepackage{geometry}

% Include hyperref last.
\usepackage{hyperref}
% Fix anchor placement for figures with captions.
\usepackage{hypcap}% it must be loaded after hyperref.
% Set up styles of URL: it should be placed after hyperref.
\urlstyle{same}

\addto\captionsenglish{\renewcommand{\figurename}{Fig.}}
\addto\captionsenglish{\renewcommand{\tablename}{Table}}
\addto\captionsenglish{\renewcommand{\literalblockname}{Listing}}

\addto\captionsenglish{\renewcommand{\literalblockcontinuedname}{continued from previous page}}
\addto\captionsenglish{\renewcommand{\literalblockcontinuesname}{continues on next page}}

\addto\extrasenglish{\def\pageautorefname{page}}

\setcounter{tocdepth}{1}



\title{Hazma Documentation}
\date{Jan 24, 2018}
\release{1.2}
\author{Logan A. Morrison et. al.}
\newcommand{\sphinxlogo}{\vbox{}}
\renewcommand{\releasename}{Release}
\makeindex

\begin{document}

\maketitle
\sphinxtableofcontents
\phantomsection\label{\detokenize{index::doc}}


Contents:


\chapter{Description}
\label{\detokenize{description::doc}}\label{\detokenize{description:description}}\label{\detokenize{description:welcome-to-hazma-s-documentation}}

\section{Introduction}
\label{\detokenize{description:introduction}}
This package is used to compute the gamma ray spectra \(\dfrac{dN}{dE}\) for light particles, such as, pions, kaon, electrons and muons, in an energy regime where the mass effects are important, i.e. is the MeV energy range. The code has been written in python/cython.


\section{Decay spectra}
\label{\detokenize{description:decay-spectra}}
In this section, we describe how the radiative decay spectra are computed for the muon, charged pion and neutral pion.


\subsection{Muon}
\label{\detokenize{description:muon}}
The dominant contribution to the radiative decay of the muon comes from \(\mu^{\pm}\to e^{\pm}\nu\bar{\nu}\gamma\). The unpolarized differential branching fraction of this decay mode in the \sphinxstyleemphasis{muon rest frame} can be written as
{[}1{]}
\begin{equation*}
\begin{split}\dfrac{dB}{dy \ d\cos\theta_{\gamma}^{R}} = \dfrac{1}{y}
\dfrac{\alpha}{72\pi}(1-y)\left[
12\left(3 - 2y(1-y)^2\right)\log\left(\dfrac{1-y}{r}\right)
+ y(1-y)(46 - 55y) - 102\right]\end{split}
\end{equation*}
where \(r = (m_{e}/m_{\mu})^2\), \(0 \leq y = 2E_{\gamma}^{R\mu}/m_{\mu} \leq 1 - r\), (\(E_{\gamma}^{R\mu}\) is the energy of the photon in the muon rest frame) and \(\theta_{\gamma}^{R}\) is the angle the photon makes with respect to some axis in the muon rest frame.  In order to obtain the decay spectrum in the laboratory frame, we need to boost the above spectrum. In other words, we need to change variables from the gamma ray energy and angle in the muon rest frame to those in the lab frame. We then integrate over the angle to compute \(dN/dE_{\gamma}\). The Jacobian for this change of variables is
\begin{equation*}
\begin{split}J = \dfrac{1}{2\gamma(1-\beta\cos\theta_{\gamma}^{L})}\end{split}
\end{equation*}
where the boost parameters are
\begin{equation*}
\begin{split}\gamma = E_{\mu} / m_{\mu}, \qquad \beta = \sqrt{1 - \left(\dfrac{m_{\mu}}{E_{\mu}}\right)^2}\end{split}
\end{equation*}
Integrating over angles yields the gamma ray spectrum in the lab frame:
\begin{equation*}
\begin{split}\dfrac{dN}{dE_{\gamma}^{L}} =
\int_{-1}^{1}d\cos\theta_{\gamma}^{L}
\dfrac{1}{2\gamma(1-\beta\cos\theta_{\gamma}^{L})}
\dfrac{dB}{dE_{\gamma}^{R\mu}}\end{split}
\end{equation*}
\noindent{\hspace*{\fill}\sphinxincludegraphics[width=400\sphinxpxdimen,height=400\sphinxpxdimen]{{muon_decay_spectra}.png}\hspace*{\fill}}


\subsection{Charged Pion}
\label{\detokenize{description:charged-pion}}
To compute the gamma ray spectrum from a charged pion, one considers to possible decay modes. These decay modes are \(\pi^{\pm} \to \mu^{\pm}\nu_{\mu}\gamma\) and \(\pi^{\pm} \to \mu^{\pm}\nu_{\mu} \to e^{\pm}\nu_{\mu}\nu_{\mu}\nu_{e}\gamma\). To compute the gamma ray spectrum from the first decay mode, one uses results from {[}2{]}. It turns out that the spectrum from this decay mode is roughly a factor of 100 times smaller than the spectrum from the second decay mode. We thus ignore the contributions from \(\pi^{\pm} \to \mu^{\pm}\nu_{\mu}\gamma\).

To compute the \(\gamma\)-ray spectrum from \(\pi^{\pm} \to \mu^{\pm}\nu_{\mu} \to e^{\pm}\nu_{\mu}\nu_{\mu}\nu_{e}\gamma\), we first take the muon decay spectra (see section on muon decay spectra) and boost the muon into the pion rest frame use the following:
\begin{equation*}
\begin{split}\gamma_{1} = E_{R\mu}/m_{\mu} \qquad
\beta_{1} = \sqrt{1-\left(\dfrac{m_{\mu}}{E_{R\mu}}\right)^2} \qquad  E_{R\mu} = \dfrac{m_{\pi}^2 - m_{\mu}^2}{m_{\pi}^2 + m_{\mu}^2}\end{split}
\end{equation*}
where \(E_{R\mu}\) is the energy of the muon in the pion rest frame. The photon spectrum in the charged pion rest frame, \(dN/dE_{\gamma}^{R\pi}\), is obtain by integrating the differential branching ratio times a Jacobian factor \(1/2\gamma_{1}(1-\beta_{1}\cos\theta_{\gamma}^{R\pi})\) over the
angle the photon makes with the muon. Once this integration is completed, one then boosts into the laboratory frame of reference. The steps are nearly identical to boosting from the muon rest frame to the pion rest frame. The only thing that changes in the boost factor and the Jacobian. In going from the charged pion rest frame to the laboratory frame, the Jacobian and boost factor are
\begin{equation*}
\begin{split}J = \dfrac{1}{2\gamma_{2}(1-\beta_{2}\cos\theta_{\gamma}^{L})} \qquad
\gamma_{2} = E_{\pi} / m_{\pi} \qquad
\beta_{2} = \sqrt{1 - \left(\dfrac{m_{\mu}}{E_{\pi}}\right)^2}\end{split}
\end{equation*}
The gamma-ray spectrum in the laboratory frame will thus be
\begin{equation*}
\begin{split}\dfrac{dN}{dE_{\gamma}^{L}} = \int_{-1}^{1} d\cos\theta_{\gamma}^{L} \dfrac{1}{2\gamma_{2}(1-\beta_{2}\cos\theta_{\gamma}^{L})} \times
\left(\int_{-1}^{1}d\cos\theta_{\gamma}^{R\pi}
\dfrac{1}{2\gamma_{1}(1-\beta_{1}\cos\theta_{\gamma}^{L})}
\dfrac{dB}{dE_{\gamma}^{R\mu}}
\right)\end{split}
\end{equation*}
where
\begin{equation*}
\begin{split}E_{\gamma}^{R\mu} = \gamma_{1} E_{\gamma}^{R\pi}\left(1 - \beta_{1}\cos\theta_{\gamma}^{R\pi}\right)\end{split}
\end{equation*}
and
\begin{equation*}
\begin{split}E_{\gamma}^{R\pi} = \gamma_{2} E_{\gamma}^{L}\left(1 - \beta_{2}\cos\theta_{\gamma}^{L}\right)\end{split}
\end{equation*}
The limits on the photon energy are given by
\begin{equation*}
\begin{split}0 \leq E_{\gamma}^{L} \leq \dfrac{m_{\mu}^2 - m_{e}^2}{2m_{\mu}}
\gamma_{1}\gamma_{2}(1+\beta_{1})(1+\beta_{2})\end{split}
\end{equation*}
\noindent{\hspace*{\fill}\sphinxincludegraphics[width=400\sphinxpxdimen,height=400\sphinxpxdimen]{{charged_pion_decay_spectrum}.png}\hspace*{\fill}}


\subsection{Neutral Pion}
\label{\detokenize{description:neutral-pion}}
The dominant decay mode of the neutral pion is \(\pi^{0}\to\gamma\gamma\). In the laboratory frame, the spectrum is
\begin{equation*}
\begin{split}\dfrac{dN}{dE_{\gamma}} = \dfrac{2}{m_{\pi}\gamma\beta}\end{split}
\end{equation*}

\section{Final State Radiation}
\label{\detokenize{description:final-state-radiation}}
Along with computing decay spectra, hazma is able to compute final state radiation spectra from decays of off-shell mediators (scalar, psuedo-scalar, vector or axial-vector.) The relavent diagrams for such processes are


\begin{savenotes}\sphinxattablestart
\centering
\begin{tabular}[t]{|*{2}{\X{1}{2}|}}
\hline
\begin{sphinxfigure-in-table}
\centering

\noindent\sphinxincludegraphics{{scalar-fsr}.png}
\end{sphinxfigure-in-table}\relax
\begin{enumerate}
\item {} 
Scalar mediator

\end{enumerate}
&\begin{sphinxfigure-in-table}
\centering

\noindent\sphinxincludegraphics{{vector-fsr}.png}
\end{sphinxfigure-in-table}\relax
\begin{enumerate}
\setcounter{enumi}{1}
\item {} 
Vector mediator

\end{enumerate}
\\
\hline
\end{tabular}
\par
\sphinxattableend\end{savenotes}

Computing the matrix elements squared of these diagrams (including diagrams with the photon attached to the other fermion leg) and integrating over all variables except the photon energy yields \(d\sigma(M^*\to\mu^{+}\mu^{-}\gamma)/dE\). To compute \(dN/dE\), we divide \(d\sigma(M^*\to\mu^{+}\mu^{-}\gamma)/dE\) by \(\sigma(M^*\to\mu^{+}\mu^{-})\).

\noindent{\hspace*{\fill}\sphinxincludegraphics[width=800\sphinxpxdimen,height=800\sphinxpxdimen]{{muon_fsr}.png}\hspace*{\fill}}


\subsection{References}
\label{\detokenize{description:references}}

\chapter{Gamma Ray Spectra Generator (hazma.gamma\_ray)}
\label{\detokenize{gamma_ray:gamma-ray-spectra-generator-hazma-gamma-ray}}\label{\detokenize{gamma_ray::doc}}

\section{Description}
\label{\detokenize{gamma_ray:description}}
Sub-package for generating gamma ray spectra given a multi-particle final state.

Hazma includes two different methods for computing gamma ray spectra. The first is done by specifying the final state of a process. Doing so, the particle decay spectra are then computed. The second method \sphinxcode{gamma\_ray\_rambo} takes in the tree-level and radiative squared matrix elements and runs a Monte-Carlo to generate the gamma ray spectra.


\section{Functions}
\label{\detokenize{gamma_ray:functions}}

\begin{savenotes}\sphinxattablestart
\centering
\begin{tabulary}{\linewidth}[t]{|T|T|}
\hline

Generate spectrum from builtin functions
&
\DUrole{xref,std,std-ref}{func\_gamma\_ray}
\\
\hline
Generate spectrum using Monte Carlo
&
\DUrole{xref,std,std-ref}{func\_gamma\_ray\_rambo}
\\
\hline
\end{tabulary}
\par
\sphinxattableend\end{savenotes}


\chapter{RAMBO (hazma.rambo)}
\label{\detokenize{rambo::doc}}\label{\detokenize{rambo:rambo-hazma-rambo}}

\section{Description}
\label{\detokenize{rambo:description}}
Sub-package for generating phases space points and computing phase space integrals using a Monte Carlo algorithm called RAMBO.


\section{Functions}
\label{\detokenize{rambo:functions}}

\begin{savenotes}\sphinxattablestart
\centering
\begin{tabulary}{\linewidth}[t]{|T|T|}
\hline

Computing annihilation
cross sections
&
\DUrole{xref,std,std-ref}{func\_compute\_annihilation\_cross\_section}
\\
\hline
Computing decay widths
&
\DUrole{xref,std,std-ref}{func\_compute\_decay\_width}
\\
\hline
Computing energy
histograms for final
state particles
&
\DUrole{xref,std,std-ref}{func\_generate\_energy\_histogram}
\\
\hline
Compute a single
relativistic phase
space point
&
\DUrole{xref,std,std-ref}{func\_generate\_phase\_space\_point}
\\
\hline
Compute many
relativistic phase
space points
&
\DUrole{xref,std,std-ref}{func\_generate\_phase\_space}
\\
\hline
\end{tabulary}
\par
\sphinxattableend\end{savenotes}


\chapter{Decay (hazma.decay)}
\label{\detokenize{decay:decay-hazma-decay}}\label{\detokenize{decay::doc}}

\section{Description}
\label{\detokenize{decay:description}}

\section{Functions}
\label{\detokenize{decay:functions}}

\begin{savenotes}\sphinxattablestart
\centering
\begin{tabulary}{\linewidth}[t]{|T|T|}
\hline

Muon
&
\DUrole{xref,std,std-ref}{func\_muon\_decay}
\\
\hline
Neutral Pion
&
\DUrole{xref,std,std-ref}{func\_neutral\_pion\_decay}
\\
\hline
Charged Pion
&
\DUrole{xref,std,std-ref}{func\_charged\_pion\_decay}
\\
\hline
Short Kaon
&
\DUrole{xref,std,std-ref}{func\_short\_kaon\_decay}
\\
\hline
Long Kaon
&
\DUrole{xref,std,std-ref}{func\_long\_kaon\_decay}
\\
\hline
Charged Kaon
&
\DUrole{xref,std,std-ref}{func\_charged\_kaon\_decay}
\\
\hline
\end{tabulary}
\par
\sphinxattableend\end{savenotes}


\chapter{Indices and tables}
\label{\detokenize{index:indices-and-tables}}\begin{itemize}
\item {} 
\DUrole{xref,std,std-ref}{genindex}

\item {} 
\DUrole{xref,std,std-ref}{modindex}

\item {} 
\DUrole{xref,std,std-ref}{search}

\end{itemize}



\renewcommand{\indexname}{Index}
\printindex
\end{document}