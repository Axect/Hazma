%% Generated by Sphinx.
\def\sphinxdocclass{report}
\documentclass[letterpaper,10pt,english]{sphinxmanual}
\ifdefined\pdfpxdimen
   \let\sphinxpxdimen\pdfpxdimen\else\newdimen\sphinxpxdimen
\fi \sphinxpxdimen=.75bp\relax

\usepackage[utf8]{inputenc}
\ifdefined\DeclareUnicodeCharacter
 \ifdefined\DeclareUnicodeCharacterAsOptional
  \DeclareUnicodeCharacter{"00A0}{\nobreakspace}
  \DeclareUnicodeCharacter{"2500}{\sphinxunichar{2500}}
  \DeclareUnicodeCharacter{"2502}{\sphinxunichar{2502}}
  \DeclareUnicodeCharacter{"2514}{\sphinxunichar{2514}}
  \DeclareUnicodeCharacter{"251C}{\sphinxunichar{251C}}
  \DeclareUnicodeCharacter{"2572}{\textbackslash}
 \else
  \DeclareUnicodeCharacter{00A0}{\nobreakspace}
  \DeclareUnicodeCharacter{2500}{\sphinxunichar{2500}}
  \DeclareUnicodeCharacter{2502}{\sphinxunichar{2502}}
  \DeclareUnicodeCharacter{2514}{\sphinxunichar{2514}}
  \DeclareUnicodeCharacter{251C}{\sphinxunichar{251C}}
  \DeclareUnicodeCharacter{2572}{\textbackslash}
 \fi
\fi
\usepackage{cmap}
\usepackage[T1]{fontenc}
\usepackage{amsmath,amssymb,amstext}
\usepackage{babel}
\usepackage{times}
\usepackage[Bjarne]{fncychap}
\usepackage[dontkeepoldnames]{sphinx}

\usepackage{geometry}

% Include hyperref last.
\usepackage{hyperref}
% Fix anchor placement for figures with captions.
\usepackage{hypcap}% it must be loaded after hyperref.
% Set up styles of URL: it should be placed after hyperref.
\urlstyle{same}

\addto\captionsenglish{\renewcommand{\figurename}{Fig.}}
\addto\captionsenglish{\renewcommand{\tablename}{Table}}
\addto\captionsenglish{\renewcommand{\literalblockname}{Listing}}

\addto\captionsenglish{\renewcommand{\literalblockcontinuedname}{continued from previous page}}
\addto\captionsenglish{\renewcommand{\literalblockcontinuesname}{continues on next page}}

\addto\extrasenglish{\def\pageautorefname{page}}

\setcounter{tocdepth}{1}



\title{Hazma Documentation}
\date{Dec 20, 2017}
\release{1.0}
\author{Logan A. Morrison et. al.}
\newcommand{\sphinxlogo}{\vbox{}}
\renewcommand{\releasename}{Release}
\makeindex

\begin{document}

\maketitle
\sphinxtableofcontents
\phantomsection\label{\detokenize{index::doc}}


Contents:


\chapter{Description}
\label{\detokenize{description::doc}}\label{\detokenize{description:description}}\label{\detokenize{description:welcome-to-hazma-s-documentation}}

\section{Introduction}
\label{\detokenize{description:introduction}}
This package is used to compute the gamma ray spectra \(\dfrac{dN}{dE}\) for light particles, such as, pions, kaon, electrons and muons, in an energy regime where the mass effects are important, i.e. is the MeV energy range. The code has been written in python/cython.


\section{Decay spectra}
\label{\detokenize{description:decay-spectra}}
In this section, we describe how the radiative decay spectra are computed for the muon, charged pion and neutral pion.


\subsection{Muon}
\label{\detokenize{description:muon}}
The dominant contribution to the radiative decay of the muon comes from \(\mu^{\pm}\to e^{\pm}\nu\bar{\nu}\gamma\). The unpolarized differential branching fraction of this decay mode in the \sphinxstyleemphasis{muon rest frame} can be written as
{[}1{]}
\begin{equation*}
\begin{split}\dfrac{dB}{dy \ d\cos\theta_{\gamma}^{R}} = \dfrac{1}{y}
\dfrac{\alpha}{72\pi}(1-y)\left[
12\left(3 - 2y(1-y)^2\right)\log\left(\dfrac{1-y}{r}\right)
+ y(1-y)(46 - 55y) - 102\right]\end{split}
\end{equation*}
where \(r = (m_{e}/m_{\mu})^2\), \(0 \leq y = 2E_{\gamma}^{R\mu}/m_{\mu} \leq 1 - r\), (\(E_{\gamma}^{R\mu}\) is the energy of the photon in the muon rest frame) and \(\theta_{\gamma}^{R}\) is the angle the photon makes with respect to some axis in the muon rest frame.  In order to obtain the decay spectrum in the laboratory frame, we need to boost the above spectrum. In other words, we need to change variables from the gamma ray energy and angle in the muon rest frame to those in the lab frame. We then integrate over the angle to compute \(dN/dE_{\gamma}\). The Jacobian for this change of variables is
\begin{equation*}
\begin{split}J = \dfrac{1}{2\gamma(1-\beta\cos\theta_{\gamma}^{L})}\end{split}
\end{equation*}
where the boost parameters are
\begin{equation*}
\begin{split}\gamma = E_{\mu} / m_{\mu}, \qquad \beta = \sqrt{1 - \left(\dfrac{m_{\mu}}{E_{\mu}}\right)^2}\end{split}
\end{equation*}
Integrating over angles yields the gamma ray spectrum in the lab frame:
\begin{equation*}
\begin{split}\dfrac{dN}{dE_{\gamma}^{L}} =
\int_{-1}^{1}d\cos\theta_{\gamma}^{L}
\dfrac{1}{2\gamma(1-\beta\cos\theta_{\gamma}^{L})}
\dfrac{dB}{dE_{\gamma}^{R\mu}}\end{split}
\end{equation*}
\noindent{\hspace*{\fill}\sphinxincludegraphics[width=400\sphinxpxdimen,height=400\sphinxpxdimen]{{muon_decay_spectra}.png}\hspace*{\fill}}


\subsection{Charged Pion}
\label{\detokenize{description:charged-pion}}
To compute the gamma ray spectrum from a charged pion, one considers to possible decay modes. These decay modes are \(\pi^{\pm} \to \mu^{\pm}\nu_{\mu}\gamma\) and \(\pi^{\pm} \to \mu^{\pm}\nu_{\mu} \to e^{\pm}\nu_{\mu}\nu_{\mu}\nu_{e}\gamma\). To compute the gamma ray spectrum from the first decay mode, one uses results from {[}2{]}. It turns out that the spectrum from this decay mode is roughly a factor of 100 times smaller than the spectrum from the second decay mode. We thus ignore the contributions from \(\pi^{\pm} \to \mu^{\pm}\nu_{\mu}\gamma\).

To compute the \(\gamma\)-ray spectrum from \(\pi^{\pm} \to \mu^{\pm}\nu_{\mu} \to e^{\pm}\nu_{\mu}\nu_{\mu}\nu_{e}\gamma\), we first take the muon decay spectra (see section on muon decay spectra) and boost the muon into the pion rest frame use the following:
\begin{equation*}
\begin{split}\gamma_{1} = E_{R\mu}/m_{\mu} \qquad
\beta_{1} = \sqrt{1-\left(\dfrac{m_{\mu}}{E_{R\mu}}\right)^2} \qquad  E_{R\mu} = \dfrac{m_{\pi}^2 - m_{\mu}^2}{m_{\pi}^2 + m_{\mu}^2}\end{split}
\end{equation*}
where \(E_{R\mu}\) is the energy of the muon in the pion rest frame. The photon spectrum in the charged pion rest frame, \(dN/dE_{\gamma}^{R\pi}\), is obtain by integrating the differential branching ratio times a Jacobian factor \(1/2\gamma_{1}(1-\beta_{1}\cos\theta_{\gamma}^{R\pi})\) over the
angle the photon makes with the muon. Once this integration is completed, one then boosts into the laboratory frame of reference. The steps are nearly identical to boosting from the muon rest frame to the pion rest frame. The only thing that changes in the boost factor and the Jacobian. In going from the charged pion rest frame to the laboratory frame, the Jacobian and boost factor are
\begin{equation*}
\begin{split}J = \dfrac{1}{2\gamma_{2}(1-\beta_{2}\cos\theta_{\gamma}^{L})} \qquad
\gamma_{2} = E_{\pi} / m_{\pi} \qquad
\beta_{2} = \sqrt{1 - \left(\dfrac{m_{\mu}}{E_{\pi}}\right)^2}\end{split}
\end{equation*}
The gamma-ray spectrum in the laboratory frame will thus be
\begin{equation*}
\begin{split}\dfrac{dN}{dE_{\gamma}^{L}} = \int_{-1}^{1} d\cos\theta_{\gamma}^{L} \dfrac{1}{2\gamma_{2}(1-\beta_{2}\cos\theta_{\gamma}^{L})} \times
\left(\int_{-1}^{1}d\cos\theta_{\gamma}^{R\pi}
\dfrac{1}{2\gamma_{1}(1-\beta_{1}\cos\theta_{\gamma}^{L})}
\dfrac{dB}{dE_{\gamma}^{R\mu}}
\right)\end{split}
\end{equation*}
where
\begin{equation*}
\begin{split}E_{\gamma}^{R\mu} = \gamma_{1} E_{\gamma}^{R\pi}\left(1 - \beta_{1}\cos\theta_{\gamma}^{R\pi}\right)\end{split}
\end{equation*}
and
\begin{equation*}
\begin{split}E_{\gamma}^{R\pi} = \gamma_{2} E_{\gamma}^{L}\left(1 - \beta_{2}\cos\theta_{\gamma}^{L}\right)\end{split}
\end{equation*}
The limits on the photon energy are given by
\begin{equation*}
\begin{split}0 \leq E_{\gamma}^{L} \leq \dfrac{m_{\mu}^2 - m_{e}^2}{2m_{\mu}}
\gamma_{1}\gamma_{2}(1+\beta_{1})(1+\beta_{2})\end{split}
\end{equation*}
\noindent{\hspace*{\fill}\sphinxincludegraphics[width=400\sphinxpxdimen,height=400\sphinxpxdimen]{{charged_pion_decay_spectrum}.png}\hspace*{\fill}}


\subsection{Neutral Pion}
\label{\detokenize{description:neutral-pion}}
The dominant decay mode of the neutral pion is \(\pi^{0}\to\gamma\gamma\). In the laboratory frame, the spectrum is
\begin{equation*}
\begin{split}\dfrac{dN}{dE_{\gamma}} = \dfrac{2}{m_{\pi}\gamma\beta}\end{split}
\end{equation*}

\section{Final State Radiation}
\label{\detokenize{description:final-state-radiation}}
Along with computing decay spectra, hazma is able to compute final state radiation spectra from decays of off-shell mediators (scalar, psuedo-scalar, vector or axial-vector.) The relavent diagrams for such processes are


\begin{savenotes}\sphinxattablestart
\centering
\begin{tabular}[t]{|*{2}{\X{1}{2}|}}
\hline
\begin{sphinxfigure-in-table}
\centering

\noindent\sphinxincludegraphics{{scalar-fsr}.png}
\end{sphinxfigure-in-table}\relax
\begin{enumerate}
\item {} 
Scalar mediator

\end{enumerate}
&\begin{sphinxfigure-in-table}
\centering

\noindent\sphinxincludegraphics{{vector-fsr}.png}
\end{sphinxfigure-in-table}\relax
\begin{enumerate}
\setcounter{enumi}{1}
\item {} 
Vector mediator

\end{enumerate}
\\
\hline
\end{tabular}
\par
\sphinxattableend\end{savenotes}

Computing the matrix elements squared of these diagrams (including diagrams with the photon attached to the other fermion leg) and integrating over all variables except the photon energy yields \(d\sigma(M^*\to\mu^{+}\mu^{-}\gamma)/dE\). To compute \(dN/dE\), we divide \(d\sigma(M^*\to\mu^{+}\mu^{-}\gamma)/dE\) by \(\sigma(M^*\to\mu^{+}\mu^{-})\).

\noindent{\hspace*{\fill}\sphinxincludegraphics[width=800\sphinxpxdimen,height=800\sphinxpxdimen]{{muon_fsr}.png}\hspace*{\fill}}


\subsection{References}
\label{\detokenize{description:references}}

\chapter{Modules}
\label{\detokenize{modules:modules}}\label{\detokenize{modules::doc}}
The main modules of Hazma are the particle modules (electron, muon, charged pion, neutral pion, charged kaon and the neutral kaons) and the \sphinxtitleref{gamma\_ray} module. Each of the particle modules have two functions \sphinxcode{decay\_spectra} and \sphinxcode{fsr} which produce the gamma ray spectra from radiative decays and final state radiation, respectively.


\section{Gamma Ray (\sphinxstyleliteralintitle{hazma.gamma\_ray})}
\label{\detokenize{modules:module-hazma.gamma_ray}}\label{\detokenize{modules:gamma-ray-hazma-gamma-ray}}\index{hazma.gamma\_ray (module)}
Module for computing gamma ray spectra from a many-particle final state.
\begin{itemize}
\item {} 
author : Logan Morrison and Adam Coogan

\item {} 
date : December 2017

\end{itemize}
\index{gamma\_ray() (in module hazma.gamma\_ray)}

\begin{fulllineitems}
\phantomsection\label{\detokenize{modules:hazma.gamma_ray.gamma_ray}}\pysiglinewithargsret{\sphinxcode{hazma.gamma\_ray.}\sphinxbfcode{gamma\_ray}}{\emph{particles}, \emph{cme}, \emph{eng\_gams}, \emph{mat\_elem\_sqrd=\textless{}function \textless{}lambda\textgreater{}\textgreater{}}, \emph{num\_ps\_pts=1000}, \emph{num\_bins=25}}{}
Returns total gamma ray spectrum from a set of particles.
\begin{quote}

Blah Blah
\end{quote}
\begin{quote}\begin{description}
\item[{Parameters}] \leavevmode
\sphinxstylestrong{particles} : np.ndarray
\begin{quote}
\begin{quote}

List of particle names. Availible particles are ‘muon’, ‘electron’
‘charged\_pion’, ‘neutral pion’, ‘charged\_kaon’, ‘long\_kaon’,
‘short\_kaon’
\end{quote}
\begin{description}
\item[{cme}] \leavevmode{[}double{]}
Center of mass energy of the final state in MeV.

\item[{eng\_gams}] \leavevmode{[}np.ndarray{[}double, ndim=1{]}{]}
List of gamma ray energies in MeV to evaluate spectra at.

\item[{mat\_elem\_sqrd}] \leavevmode{[}double({\color{red}\bfseries{}*}func)(np.ndarray, ){]}
Function for the matrix element squared of the proccess. Must be a
function taking in a list of four momenta of size (num\_fsp, 4).
Default value is a flat matrix element.

\item[{num\_ps\_pts}] \leavevmode{[}int \{1000\}, optional{]}
Number of phase space points to use.

\item[{num\_bins}] \leavevmode{[}int \{25\}, optional{]}
Number of bins to use.

\end{description}
\end{quote}

\item[{Returns}] \leavevmode
\sphinxstylestrong{spec} : np.ndarray
\begin{quote}

Total gamma ray spectrum from all final state particles.
\end{quote}

\end{description}\end{quote}
\begin{description}
\item[{rac\{dN\}\{dE\}(E\_\{gamma\}) =}] \leavevmode
sum\_\{i,j\}P\_\{i\}(E\_\{j\})

\end{description}

rac\{dN\_i\}\{dE\}(E\_\{gamma\}, E\_\{j\})
\begin{quote}

where \(i\) runs over the final state particles, \(j\) runs over
energies sampled from probability distributions. \(P_{i}(E_{j})\) is
the probability that particle \(i\) has energy \(E_{j}\). The
probabilities are computed using \sphinxtitleref{hazma.phase\_space\_generator.rambo}. The
total number of energies used is \sphinxtitleref{num\_bins}.
\end{quote}
\paragraph{Examples}

Example of generating a spectrum from a muon, charged kaon and long kaon
with total energy of 5000 MeV.

\begin{sphinxVerbatim}[commandchars=\\\{\}]
\PYG{g+gp}{\PYGZgt{}\PYGZgt{}\PYGZgt{} }\PYG{k+kn}{from} \PYG{n+nn}{hazma}\PYG{n+nn}{.}\PYG{n+nn}{gamma\PYGZus{}ray} \PYG{k}{import} \PYG{n}{gamma\PYGZus{}ray}
\PYG{g+gp}{\PYGZgt{}\PYGZgt{}\PYGZgt{} }\PYG{k+kn}{import} \PYG{n+nn}{numpy} \PYG{k}{as} \PYG{n+nn}{np}
\PYG{g+go}{\PYGZgt{}\PYGZgt{}\PYGZgt{}}
\PYG{g+gp}{\PYGZgt{}\PYGZgt{}\PYGZgt{} }\PYG{n}{particles} \PYG{o}{=} \PYG{n}{np}\PYG{o}{.}\PYG{n}{array}\PYG{p}{(}\PYG{p}{[}\PYG{l+s+s1}{\PYGZsq{}}\PYG{l+s+s1}{muon}\PYG{l+s+s1}{\PYGZsq{}}\PYG{p}{,} \PYG{l+s+s1}{\PYGZsq{}}\PYG{l+s+s1}{charged\PYGZus{}kaon}\PYG{l+s+s1}{\PYGZsq{}}\PYG{p}{,} \PYG{l+s+s1}{\PYGZsq{}}\PYG{l+s+s1}{long\PYGZus{}kaon}\PYG{l+s+s1}{\PYGZsq{}}\PYG{p}{]}\PYG{p}{)}
\PYG{g+gp}{\PYGZgt{}\PYGZgt{}\PYGZgt{} }\PYG{n}{cme} \PYG{o}{=} \PYG{l+m+mf}{5000.}
\PYG{g+gp}{\PYGZgt{}\PYGZgt{}\PYGZgt{} }\PYG{n}{eng\PYGZus{}gams} \PYG{o}{=} \PYG{n}{np}\PYG{o}{.}\PYG{n}{logspace}\PYG{p}{(}\PYG{l+m+mf}{0.}\PYG{p}{,} \PYG{n}{np}\PYG{o}{.}\PYG{n}{log10}\PYG{p}{(}\PYG{n}{cme}\PYG{p}{)}\PYG{p}{,} \PYG{n}{num}\PYG{o}{=}\PYG{l+m+mi}{200}\PYG{p}{,} \PYG{n}{dtype}\PYG{o}{=}\PYG{n}{np}\PYG{o}{.}\PYG{n}{float64}\PYG{p}{)}
\PYG{g+go}{\PYGZgt{}\PYGZgt{}\PYGZgt{}}
\PYG{g+gp}{\PYGZgt{}\PYGZgt{}\PYGZgt{} }\PYG{n}{spec} \PYG{o}{=} \PYG{n}{gamma\PYGZus{}ray}\PYG{p}{(}\PYG{n}{particles}\PYG{p}{,} \PYG{n}{cme}\PYG{p}{,} \PYG{n}{eng\PYGZus{}gams}\PYG{p}{)}
\end{sphinxVerbatim}

\end{fulllineitems}



\section{Muon (\sphinxstyleliteralintitle{hazma.muon})}
\label{\detokenize{modules:muon-hazma-muon}}\label{\detokenize{modules:module-hazma.muon}}\index{hazma.muon (module)}
Module for computing gamma ray spectra from a muon.

@author - Logan Morrison and Adam Coogan
@date - December 2017
\index{decay\_spectra() (in module hazma.muon)}

\begin{fulllineitems}
\phantomsection\label{\detokenize{modules:hazma.muon.decay_spectra}}\pysiglinewithargsret{\sphinxcode{hazma.muon.}\sphinxbfcode{decay\_spectra}}{\emph{eng\_gam}, \emph{eng\_mu}}{}
Compute dNdE from muon decay.

Compute dNdE from decay mu -\textgreater{} e nu nu gamma in the laborartory frame given
a gamma ray engergy of \sphinxcode{eng\_gam} and muon energy of \sphinxcode{eng\_mu}.
\begin{quote}\begin{description}
\item[{Parameters}] \leavevmode
\sphinxstylestrong{eng\_gam} : numpy.ndarray
\begin{quote}

Gamma ray energy(ies) in laboratory frame.
\end{quote}

\sphinxstylestrong{eng\_mu} : double
\begin{quote}

Muon energy in laboratory frame.
\end{quote}

\item[{Returns}] \leavevmode
\sphinxstylestrong{spec} : numpy.ndarray
\begin{quote}

List of gamma ray spectrum values, dNdE, evaluated at \sphinxcode{eng\_gam} given
muon energy \sphinxcode{eng\_mu}.
\end{quote}

\end{description}\end{quote}
\paragraph{Examples}

Calculate spectrum for single gamma ray energy

\begin{sphinxVerbatim}[commandchars=\\\{\}]
\PYG{g+gp}{\PYGZgt{}\PYGZgt{}\PYGZgt{} }\PYG{k+kn}{from} \PYG{n+nn}{hazma} \PYG{k}{import} \PYG{n}{muon}
\PYG{g+gp}{\PYGZgt{}\PYGZgt{}\PYGZgt{} }\PYG{n}{eng\PYGZus{}gam}\PYG{p}{,} \PYG{n}{eng\PYGZus{}mu} \PYG{o}{=} \PYG{l+m+mf}{200.}\PYG{p}{,} \PYG{l+m+mf}{1000.}
\PYG{g+gp}{\PYGZgt{}\PYGZgt{}\PYGZgt{} }\PYG{n}{spec} \PYG{o}{=} \PYG{n}{muon}\PYG{o}{.}\PYG{n}{decay\PYGZus{}spectra}\PYG{p}{(}\PYG{n}{eng\PYGZus{}gam}\PYG{p}{,} \PYG{n}{eng\PYGZus{}mu}\PYG{p}{)}
\end{sphinxVerbatim}

Calculate spectrum for array of gamma ray energies

\begin{sphinxVerbatim}[commandchars=\\\{\}]
\PYG{g+gp}{\PYGZgt{}\PYGZgt{}\PYGZgt{} }\PYG{k+kn}{from} \PYG{n+nn}{hazma} \PYG{k}{import} \PYG{n}{muon}
\PYG{g+gp}{\PYGZgt{}\PYGZgt{}\PYGZgt{} }\PYG{k+kn}{import} \PYG{n+nn}{numpy} \PYG{k}{as} \PYG{n+nn}{np}
\PYG{g+gp}{\PYGZgt{}\PYGZgt{}\PYGZgt{} }\PYG{n}{eng\PYGZus{}gams} \PYG{o}{=} \PYG{n}{np}\PYG{o}{.}\PYG{n}{logspace}\PYG{p}{(}\PYG{l+m+mf}{0.0}\PYG{p}{,} \PYG{l+m+mf}{3.0}\PYG{p}{,} \PYG{n}{num}\PYG{o}{=}\PYG{l+m+mi}{200}\PYG{p}{,} \PYG{n}{dtype}\PYG{o}{=}\PYG{n+nb}{float}\PYG{p}{)}
\PYG{g+gp}{\PYGZgt{}\PYGZgt{}\PYGZgt{} }\PYG{n}{eng\PYGZus{}mu} \PYG{o}{=} \PYG{l+m+mf}{1000.}
\PYG{g+gp}{\PYGZgt{}\PYGZgt{}\PYGZgt{} }\PYG{n}{spec} \PYG{o}{=} \PYG{n}{muon}\PYG{o}{.}\PYG{n}{decay\PYGZus{}spectra}\PYG{p}{(}\PYG{n}{eng\PYGZus{}gams}\PYG{p}{,} \PYG{n}{eng\PYGZus{}mu}\PYG{p}{)}
\end{sphinxVerbatim}

\end{fulllineitems}

\index{fsr() (in module hazma.muon)}

\begin{fulllineitems}
\phantomsection\label{\detokenize{modules:hazma.muon.fsr}}\pysiglinewithargsret{\sphinxcode{hazma.muon.}\sphinxbfcode{fsr}}{\emph{eng\_gam}, \emph{cme}, \emph{mediator='scalar'}}{}
Compute muon fsr spectrum.

Compute final state radiation spectrum dN/dE from decay of an off-shell
mediator (scalar, psuedo-scalar, vector or axial-vector) into a pair of
muons.
\begin{description}
\item[{Paramaters}] \leavevmode
eng\_gam (float or np.ndarray) : Gamma ray energy(ies) in laboratory
frame.
cme (float) : Center of mass energy or mass of the off-shell mediator.
mediator (string) : Mediator type : scalar, psuedo-scalar, vector or
axial-vector.

\item[{Returns}] \leavevmode
spec (np.ndarray) : List of gamma ray spectrum values, dNdE, evaluated
at \sphinxtitleref{eng\_gams} given a center of mass energy \sphinxtitleref{cme}.

\item[{Examples}] \leavevmode
dNdE for a single gamma ray energy from scalar mediator.

\begin{sphinxVerbatim}[commandchars=\\\{\}]
\PYG{g+gp}{\PYGZgt{}\PYGZgt{}\PYGZgt{} }\PYG{k+kn}{from} \PYG{n+nn}{hazma} \PYG{k}{import} \PYG{n}{muon}
\PYG{g+gp}{\PYGZgt{}\PYGZgt{}\PYGZgt{} }\PYG{n}{eng\PYGZus{}gam}\PYG{p}{,} \PYG{n}{cme} \PYG{o}{=} \PYG{l+m+mf}{200.}\PYG{p}{,} \PYG{l+m+mf}{1000.}
\PYG{g+gp}{\PYGZgt{}\PYGZgt{}\PYGZgt{} }\PYG{n}{spec} \PYG{o}{=} \PYG{n}{muon}\PYG{o}{.}\PYG{n}{fsr}\PYG{p}{(}\PYG{n}{eng\PYGZus{}gam}\PYG{p}{,} \PYG{n}{cme}\PYG{p}{,} \PYG{l+s+s1}{\PYGZsq{}}\PYG{l+s+s1}{scalar}\PYG{l+s+s1}{\PYGZsq{}}\PYG{p}{)}
\end{sphinxVerbatim}

dNdE for list of gamma ray energies from vector mediator.

\begin{sphinxVerbatim}[commandchars=\\\{\}]
\PYG{g+gp}{\PYGZgt{}\PYGZgt{}\PYGZgt{} }\PYG{k+kn}{from} \PYG{n+nn}{hazma} \PYG{k}{import} \PYG{n}{muon}
\PYG{g+gp}{\PYGZgt{}\PYGZgt{}\PYGZgt{} }\PYG{n}{eng\PYGZus{}gams} \PYG{o}{=} \PYG{n}{np}\PYG{o}{.}\PYG{n}{logspace}\PYG{p}{(}\PYG{l+m+mf}{0.0}\PYG{p}{,} \PYG{l+m+mf}{3.0}\PYG{p}{,} \PYG{n}{num}\PYG{o}{=}\PYG{l+m+mi}{1000}\PYG{p}{,} \PYG{n}{dtype}\PYG{o}{=}\PYG{n+nb}{float}\PYG{p}{)}
\PYG{g+gp}{\PYGZgt{}\PYGZgt{}\PYGZgt{} }\PYG{n}{cme} \PYG{o}{=} \PYG{l+m+mf}{1000.}
\PYG{g+gp}{\PYGZgt{}\PYGZgt{}\PYGZgt{} }\PYG{n}{spec} \PYG{o}{=} \PYG{n}{muon}\PYG{o}{.}\PYG{n}{fsr}\PYG{p}{(}\PYG{n}{eng\PYGZus{}gams}\PYG{p}{,} \PYG{n}{cme}\PYG{p}{,} \PYG{l+s+s1}{\PYGZsq{}}\PYG{l+s+s1}{scalar}\PYG{l+s+s1}{\PYGZsq{}}\PYG{p}{)}
\end{sphinxVerbatim}

\end{description}

\end{fulllineitems}



\section{Electron (\sphinxstyleliteralintitle{hazma.electron})}
\label{\detokenize{modules:module-hazma.electron}}\label{\detokenize{modules:electron-hazma-electron}}\index{hazma.electron (module)}
Module for computing gamma ray spectra from an electron.

@author - Logan Morrison and Adam Coogan
@date - December 2017
\index{decay\_spectra() (in module hazma.electron)}

\begin{fulllineitems}
\phantomsection\label{\detokenize{modules:hazma.electron.decay_spectra}}\pysiglinewithargsret{\sphinxcode{hazma.electron.}\sphinxbfcode{decay\_spectra}}{\emph{eng\_gam}, \emph{eng\_mu}}{}
Returns zero. Electron is stable.

\end{fulllineitems}

\index{fsr() (in module hazma.electron)}

\begin{fulllineitems}
\phantomsection\label{\detokenize{modules:hazma.electron.fsr}}\pysiglinewithargsret{\sphinxcode{hazma.electron.}\sphinxbfcode{fsr}}{\emph{eng\_gam}, \emph{cme}, \emph{mediator='scalar'}}{}
Compute electron fsr spectrum.

Compute final state radiation spectrum dN/dE from decay of an off-shell
mediator (scalar, psuedo-scalar, vector or axial-vector) into a pair of
electrons.
\begin{quote}\begin{description}
\item[{Returns}] \leavevmode
\sphinxstylestrong{spec} : (np.ndarray)
\begin{quote}

List of gamma ray spectrum values, dNdE, evaluated at \sphinxtitleref{eng\_gams}
given a center of mass energy \sphinxtitleref{cme}.
\end{quote}

\end{description}\end{quote}
\paragraph{Examples}

dNdE for a single gamma ray energy from scalar mediator.

\begin{sphinxVerbatim}[commandchars=\\\{\}]
\PYG{g+gp}{\PYGZgt{}\PYGZgt{}\PYGZgt{} }\PYG{k+kn}{from} \PYG{n+nn}{hazma} \PYG{k}{import} \PYG{n}{electron}
\PYG{g+gp}{\PYGZgt{}\PYGZgt{}\PYGZgt{} }\PYG{n}{eng\PYGZus{}gam}\PYG{p}{,} \PYG{n}{cme} \PYG{o}{=} \PYG{l+m+mf}{200.}\PYG{p}{,} \PYG{l+m+mf}{1000.}
\PYG{g+gp}{\PYGZgt{}\PYGZgt{}\PYGZgt{} }\PYG{n}{spec} \PYG{o}{=} \PYG{n}{electron}\PYG{o}{.}\PYG{n}{fsr}\PYG{p}{(}\PYG{n}{eng\PYGZus{}gam}\PYG{p}{,} \PYG{n}{cme}\PYG{p}{,} \PYG{l+s+s1}{\PYGZsq{}}\PYG{l+s+s1}{scalar}\PYG{l+s+s1}{\PYGZsq{}}\PYG{p}{)}
\end{sphinxVerbatim}

dNdE for list of gamma ray energies from vector mediator.

\begin{sphinxVerbatim}[commandchars=\\\{\}]
\PYG{g+gp}{\PYGZgt{}\PYGZgt{}\PYGZgt{} }\PYG{k+kn}{from} \PYG{n+nn}{hazma} \PYG{k}{import} \PYG{n}{electron}
\PYG{g+gp}{\PYGZgt{}\PYGZgt{}\PYGZgt{} }\PYG{n}{eng\PYGZus{}gams} \PYG{o}{=} \PYG{n}{np}\PYG{o}{.}\PYG{n}{logspace}\PYG{p}{(}\PYG{l+m+mf}{0.0}\PYG{p}{,} \PYG{l+m+mf}{3.0}\PYG{p}{,} \PYG{n}{num}\PYG{o}{=}\PYG{l+m+mi}{1000}\PYG{p}{,} \PYG{n}{dtype}\PYG{o}{=}\PYG{n+nb}{float}\PYG{p}{)}
\PYG{g+gp}{\PYGZgt{}\PYGZgt{}\PYGZgt{} }\PYG{n}{cme} \PYG{o}{=} \PYG{l+m+mf}{1000.}
\PYG{g+gp}{\PYGZgt{}\PYGZgt{}\PYGZgt{} }\PYG{n}{spec} \PYG{o}{=} \PYG{n}{electron}\PYG{o}{.}\PYG{n}{fsr}\PYG{p}{(}\PYG{n}{eng\PYGZus{}gams}\PYG{p}{,} \PYG{n}{cme}\PYG{p}{,} \PYG{l+s+s1}{\PYGZsq{}}\PYG{l+s+s1}{scalar}\PYG{l+s+s1}{\PYGZsq{}}\PYG{p}{)}
\end{sphinxVerbatim}

\end{fulllineitems}



\section{Charged Pion (\sphinxstyleliteralintitle{hazma.charged\_pion})}
\label{\detokenize{modules:module-hazma.charged_pion}}\label{\detokenize{modules:charged-pion-hazma-charged-pion}}\index{hazma.charged\_pion (module)}
Module for computing gamma ray spectra from a charged pion.

@author - Logan Morrison and Adam Coogan
@date - December 2017
\index{decay\_spectra() (in module hazma.charged\_pion)}

\begin{fulllineitems}
\phantomsection\label{\detokenize{modules:hazma.charged_pion.decay_spectra}}\pysiglinewithargsret{\sphinxcode{hazma.charged\_pion.}\sphinxbfcode{decay\_spectra}}{\emph{eng\_gam}, \emph{eng\_pi}}{}
Compute dNdE from charged pion decay.

Compute dNdE from decay pi -\textgreater{} mu nu -\textgreater{} e nu nu g in the laborartory frame
given a gamma ray engergy of \sphinxtitleref{eng\_gam} and muon energy of \sphinxtitleref{eng\_pi}.
\begin{quote}\begin{description}
\item[{Returns}] \leavevmode
\sphinxstylestrong{spec} : double np.ndarray
\begin{quote}

List of gamma ray spectrum values, dNdE, evaluated at \sphinxtitleref{eng\_gams} given
charged pion energy \sphinxtitleref{eng\_pi}.
\end{quote}

\end{description}\end{quote}
\paragraph{Examples}

Calculate spectrum for single gamma ray energy

\begin{sphinxVerbatim}[commandchars=\\\{\}]
\PYG{g+gp}{\PYGZgt{}\PYGZgt{}\PYGZgt{} }\PYG{k+kn}{from} \PYG{n+nn}{hazma} \PYG{k}{import} \PYG{n}{charged\PYGZus{}pion}
\PYG{g+gp}{\PYGZgt{}\PYGZgt{}\PYGZgt{} }\PYG{n}{eng\PYGZus{}gam}\PYG{p}{,} \PYG{n}{eng\PYGZus{}pi} \PYG{o}{=} \PYG{l+m+mf}{200.}\PYG{p}{,} \PYG{l+m+mf}{1000.}
\PYG{g+gp}{\PYGZgt{}\PYGZgt{}\PYGZgt{} }\PYG{n}{spec} \PYG{o}{=} \PYG{n}{charged\PYGZus{}pion}\PYG{o}{.}\PYG{n}{decay\PYGZus{}spectra}\PYG{p}{(}\PYG{n}{eng\PYGZus{}gam}\PYG{p}{,} \PYG{n}{eng\PYGZus{}pi}\PYG{p}{)}
\end{sphinxVerbatim}

Calculate spectrum for array of gamma ray energies

\begin{sphinxVerbatim}[commandchars=\\\{\}]
\PYG{g+gp}{\PYGZgt{}\PYGZgt{}\PYGZgt{} }\PYG{k+kn}{from} \PYG{n+nn}{hazma} \PYG{k}{import} \PYG{n}{charged\PYGZus{}pion}
\PYG{g+gp}{\PYGZgt{}\PYGZgt{}\PYGZgt{} }\PYG{k+kn}{import} \PYG{n+nn}{numpy} \PYG{k}{as} \PYG{n+nn}{np}
\PYG{g+gp}{\PYGZgt{}\PYGZgt{}\PYGZgt{} }\PYG{n}{eng\PYGZus{}gams} \PYG{o}{=} \PYG{n}{np}\PYG{o}{.}\PYG{n}{logspace}\PYG{p}{(}\PYG{l+m+mf}{0.0}\PYG{p}{,} \PYG{l+m+mf}{3.0}\PYG{p}{,} \PYG{n}{num}\PYG{o}{=}\PYG{l+m+mi}{200}\PYG{p}{,} \PYG{n}{dtype}\PYG{o}{=}\PYG{n+nb}{float}\PYG{p}{)}
\PYG{g+gp}{\PYGZgt{}\PYGZgt{}\PYGZgt{} }\PYG{n}{eng\PYGZus{}pi} \PYG{o}{=} \PYG{l+m+mf}{1000.}
\PYG{g+gp}{\PYGZgt{}\PYGZgt{}\PYGZgt{} }\PYG{n}{spec} \PYG{o}{=} \PYG{n}{charged\PYGZus{}pion}\PYG{o}{.}\PYG{n}{decay\PYGZus{}spectra}\PYG{p}{(}\PYG{n}{eng\PYGZus{}gams}\PYG{p}{,} \PYG{n}{eng\PYGZus{}pi}\PYG{p}{)}
\end{sphinxVerbatim}

\end{fulllineitems}

\index{fsr() (in module hazma.charged\_pion)}

\begin{fulllineitems}
\phantomsection\label{\detokenize{modules:hazma.charged_pion.fsr}}\pysiglinewithargsret{\sphinxcode{hazma.charged\_pion.}\sphinxbfcode{fsr}}{\emph{eng\_gam}, \emph{cme}, \emph{mediator='scalar'}}{}
NOT YET IMPLEMENTED!

\end{fulllineitems}



\section{Neutral Pion (\sphinxstyleliteralintitle{hazma.neutral\_pion})}
\label{\detokenize{modules:module-hazma.neutral_pion}}\label{\detokenize{modules:neutral-pion-hazma-neutral-pion}}\index{hazma.neutral\_pion (module)}
Module for computing gamma ray spectra from a neutral pion.

@author - Logan Morrison and Adam Coogan
@date - December 2017
\index{decay\_spectra() (in module hazma.neutral\_pion)}

\begin{fulllineitems}
\phantomsection\label{\detokenize{modules:hazma.neutral_pion.decay_spectra}}\pysiglinewithargsret{\sphinxcode{hazma.neutral\_pion.}\sphinxbfcode{decay\_spectra}}{\emph{eng\_gam}, \emph{eng\_pi}}{}
Compute dNdE from neutral pion decay.

Compute dNdE from decay pi0 -\textgreater{} gamma gamma in the laborartory frame given
a gamma ray engergy of \sphinxtitleref{eng\_gam} and neutral pion energy of \sphinxtitleref{eng\_pi}.
\begin{quote}\begin{description}
\item[{Returns}] \leavevmode
\sphinxstylestrong{spec} : np.ndarray
\begin{quote}

List of gamma ray spectrum values, dNdE, evaluated at
\sphinxtitleref{eng\_gams} given neutral pion energy \sphinxtitleref{eng\_pi}.
\end{quote}

\end{description}\end{quote}

\end{fulllineitems}

\index{fsr() (in module hazma.neutral\_pion)}

\begin{fulllineitems}
\phantomsection\label{\detokenize{modules:hazma.neutral_pion.fsr}}\pysiglinewithargsret{\sphinxcode{hazma.neutral\_pion.}\sphinxbfcode{fsr}}{\emph{eng\_gam}, \emph{cme}, \emph{mediator='scalar'}}{}
Returns zero.

\end{fulllineitems}



\section{Charged Kaon (\sphinxstyleliteralintitle{hazma.charged\_kaon})}
\label{\detokenize{modules:module-hazma.charged_kaon}}\label{\detokenize{modules:charged-kaon-hazma-charged-kaon}}\index{hazma.charged\_kaon (module)}
Module for computing gamma ray spectra from a charged kaon.
\begin{itemize}
\item {} 
author : Logan Morrison and Adam Coogan

\item {} 
date : December 2017

\end{itemize}
\index{decay\_spectra() (in module hazma.charged\_kaon)}

\begin{fulllineitems}
\phantomsection\label{\detokenize{modules:hazma.charged_kaon.decay_spectra}}\pysiglinewithargsret{\sphinxcode{hazma.charged\_kaon.}\sphinxbfcode{decay\_spectra}}{\emph{eng\_gam}, \emph{eng\_k}}{}
Compute dNdE from charged kaon decay.

Compute dNdE from decay of charged kaon through K -\textgreater{} X in the
laboratory frame given a gamma ray engergy of \sphinxtitleref{eng\_gam} and charged
kaon energy of \sphinxtitleref{eng\_k}. The decay modes impemendted are
* k -\textgreater{} mu  + nu
* k -\textgreater{} pi  + pi0
* k -\textgreater{} pi  + pi  + pi
* k -\textgreater{} pi0 + e   + nu
* k -\textgreater{} pi0 + mu  + nu
* k -\textgreater{} pi  + pi0 + pi0
\begin{quote}\begin{description}
\item[{Returns}] \leavevmode
\sphinxstylestrong{spec} : np.ndarray
\begin{quote}

List of gamma ray spectrum values, dNdE, evaluated at \sphinxtitleref{eng\_gams}
given muon energy \sphinxtitleref{eng\_mu}.
\end{quote}

\end{description}\end{quote}

\end{fulllineitems}

\index{fsr() (in module hazma.charged\_kaon)}

\begin{fulllineitems}
\phantomsection\label{\detokenize{modules:hazma.charged_kaon.fsr}}\pysiglinewithargsret{\sphinxcode{hazma.charged\_kaon.}\sphinxbfcode{fsr}}{\emph{eng\_gam}, \emph{cme}, \emph{mediator='scalar'}}{}
NOT YET IMPLEMENTED!

\end{fulllineitems}



\section{Long Kaon (\sphinxstyleliteralintitle{hazma.long\_kaon})}
\label{\detokenize{modules:long-kaon-hazma-long-kaon}}\label{\detokenize{modules:module-hazma.charged_kaon}}\index{hazma.charged\_kaon (module)}
Module for computing gamma ray spectra from a charged kaon.
\begin{itemize}
\item {} 
author : Logan Morrison and Adam Coogan

\item {} 
date : December 2017

\end{itemize}
\index{decay\_spectra() (in module hazma.charged\_kaon)}

\begin{fulllineitems}
\pysiglinewithargsret{\sphinxcode{hazma.charged\_kaon.}\sphinxbfcode{decay\_spectra}}{\emph{eng\_gam}, \emph{eng\_k}}{}
Compute dNdE from charged kaon decay.

Compute dNdE from decay of charged kaon through K -\textgreater{} X in the
laboratory frame given a gamma ray engergy of \sphinxtitleref{eng\_gam} and charged
kaon energy of \sphinxtitleref{eng\_k}. The decay modes impemendted are
* k -\textgreater{} mu  + nu
* k -\textgreater{} pi  + pi0
* k -\textgreater{} pi  + pi  + pi
* k -\textgreater{} pi0 + e   + nu
* k -\textgreater{} pi0 + mu  + nu
* k -\textgreater{} pi  + pi0 + pi0
\begin{quote}\begin{description}
\item[{Returns}] \leavevmode
\sphinxstylestrong{spec} : np.ndarray
\begin{quote}

List of gamma ray spectrum values, dNdE, evaluated at \sphinxtitleref{eng\_gams}
given muon energy \sphinxtitleref{eng\_mu}.
\end{quote}

\end{description}\end{quote}

\end{fulllineitems}

\index{fsr() (in module hazma.charged\_kaon)}

\begin{fulllineitems}
\pysiglinewithargsret{\sphinxcode{hazma.charged\_kaon.}\sphinxbfcode{fsr}}{\emph{eng\_gam}, \emph{cme}, \emph{mediator='scalar'}}{}
NOT YET IMPLEMENTED!

\end{fulllineitems}



\section{Short Kaon (\sphinxstyleliteralintitle{hazma.short\_kaon})}
\label{\detokenize{modules:short-kaon-hazma-short-kaon}}\label{\detokenize{modules:module-hazma.charged_kaon}}\index{hazma.charged\_kaon (module)}
Module for computing gamma ray spectra from a charged kaon.
\begin{itemize}
\item {} 
author : Logan Morrison and Adam Coogan

\item {} 
date : December 2017

\end{itemize}
\index{decay\_spectra() (in module hazma.charged\_kaon)}

\begin{fulllineitems}
\pysiglinewithargsret{\sphinxcode{hazma.charged\_kaon.}\sphinxbfcode{decay\_spectra}}{\emph{eng\_gam}, \emph{eng\_k}}{}
Compute dNdE from charged kaon decay.

Compute dNdE from decay of charged kaon through K -\textgreater{} X in the
laboratory frame given a gamma ray engergy of \sphinxtitleref{eng\_gam} and charged
kaon energy of \sphinxtitleref{eng\_k}. The decay modes impemendted are
* k -\textgreater{} mu  + nu
* k -\textgreater{} pi  + pi0
* k -\textgreater{} pi  + pi  + pi
* k -\textgreater{} pi0 + e   + nu
* k -\textgreater{} pi0 + mu  + nu
* k -\textgreater{} pi  + pi0 + pi0
\begin{quote}\begin{description}
\item[{Returns}] \leavevmode
\sphinxstylestrong{spec} : np.ndarray
\begin{quote}

List of gamma ray spectrum values, dNdE, evaluated at \sphinxtitleref{eng\_gams}
given muon energy \sphinxtitleref{eng\_mu}.
\end{quote}

\end{description}\end{quote}

\end{fulllineitems}

\index{fsr() (in module hazma.charged\_kaon)}

\begin{fulllineitems}
\pysiglinewithargsret{\sphinxcode{hazma.charged\_kaon.}\sphinxbfcode{fsr}}{\emph{eng\_gam}, \emph{cme}, \emph{mediator='scalar'}}{}
NOT YET IMPLEMENTED!

\end{fulllineitems}



\chapter{Usage}
\label{\detokenize{usage:usage}}\label{\detokenize{usage::doc}}
Will update soon!


\chapter{Indices and tables}
\label{\detokenize{index:indices-and-tables}}\begin{itemize}
\item {} 
\DUrole{xref,std,std-ref}{genindex}

\item {} 
\DUrole{xref,std,std-ref}{modindex}

\item {} 
\DUrole{xref,std,std-ref}{search}

\end{itemize}


\renewcommand{\indexname}{Python Module Index}
\begin{sphinxtheindex}
\def\bigletter#1{{\Large\sffamily#1}\nopagebreak\vspace{1mm}}
\bigletter{h}
\item {\sphinxstyleindexentry{hazma.charged\_kaon}}\sphinxstyleindexpageref{modules:\detokenize{module-hazma.charged_kaon}}
\item {\sphinxstyleindexentry{hazma.charged\_pion}}\sphinxstyleindexpageref{modules:\detokenize{module-hazma.charged_pion}}
\item {\sphinxstyleindexentry{hazma.electron}}\sphinxstyleindexpageref{modules:\detokenize{module-hazma.electron}}
\item {\sphinxstyleindexentry{hazma.gamma\_ray}}\sphinxstyleindexpageref{modules:\detokenize{module-hazma.gamma_ray}}
\item {\sphinxstyleindexentry{hazma.muon}}\sphinxstyleindexpageref{modules:\detokenize{module-hazma.muon}}
\item {\sphinxstyleindexentry{hazma.neutral\_pion}}\sphinxstyleindexpageref{modules:\detokenize{module-hazma.neutral_pion}}
\end{sphinxtheindex}

\renewcommand{\indexname}{Index}
\printindex
\end{document}